
\documentclass{article}
\usepackage[utf8]{inputenc}
\usepackage{relsize}
\usepackage{amsmath}
\usepackage{geometry}
\usepackage{tikz}
\usetikzlibrary{automata, positioning, arrows}
\tikzset{node distance=2.5cm, % Minimum distance between two nodes. Change if necessary.
every state/.style={ % Sets the properties for each state
semithick,
fill=gray!10},
initial text={}, % No label on start arrow
double distance=2pt, % Adjust appearance of accept states
every edge/.style={ % Sets the properties for each transition
draw,->,>=stealth', % Makes edges directed with bold arrowheads
auto,semithick}}

\geometry{
 a4paper,
 total={170mm,257mm},
 left=20mm,
 top=20mm,
}

\title{Homework 6\\[0.2em]\smaller{}CSC 445-01: Theory of Computation}
\author{Matthew Mabrey, Luke Kurlandski}
\date{\today}

\begin{document}

\maketitle

\section*{4.3}
Construct a TM S that will decide $ALL_{DFA} = \{\langle A \rangle \mid A$ is a DFA and $L(A) = \Sigma^*$\}. Inside S the machine will check:
\begin{enumerate}
    \item If all states of the input DFA $\langle A \rangle$ are final (e.g. $Q = F$), ACCEPT! 
    \item Else, REJECT!
\end{enumerate}
On any input the constructed TM $S$ now correctly: 
\begin{enumerate}
    \item Halts because $Q$ is defined as a finite set of states and $F \subseteq Q$ so our machine will never halt as it doesn't run any input and just checks equivalence of two finite sets.
    \item Accepts on $Q = F$ since the DFA will accept all strings ($\Sigma^*$).
    \item Rejects on $Q \neq F$ because there would be a non-accepting state the DFA could end in which would produce an unaccepted string, meaning it does not produce $\Sigma^*$.
\end{enumerate}
Thus the constructed TM $S$ decides $ALL_{DFA}$ because it correctly accepts or rejects and never halts on any input.

\section*{4.4}
Construct a TM S that will decide $A\varepsilon_{CFG} = \{\langle G \rangle \mid G$ is a CFG that generates $\varepsilon\}$. Inside S the machine will:
\begin{enumerate}
    \item Convert input CFG $\langle G \rangle$ to a CNF
    \item If the new CNF grammar's start state has a transition $S_0 -> \epsilon$, then accept
    \item Else, reject
\end{enumerate}

\section*{4.11}
Construct a TM S that will decide $INFINITE_{PDA} = \{\langle M \rangle \mid M$ is a PDA and $L(M)$ is infinite $\}$. Inside S the machine will:
\begin{enumerate}
    \item Non-deterministic search towards all final states
    \item If there is a loop transition AND every path that has a loop transition that includes pushing a symbol to stack has a counterpart loop that pops these characters, Accept?
    \item Else reject
\end{enumerate}

\section*{5.1}
Given a TM $R$ that we assume decides $EQ_{CFG}$ we can construct a TM $S$ that decides $A_{TM}$. Inside $S$ the machine will:

\begin{enumerate}
    \item Construct a CFG $G$ that generates $\overline{\rm L_x}$ given $A_{TM}$'s input $\langle M, w\rangle = x$ ($\overline{\rm L_x} = \Sigma^*$ iff $M$ doesn't accept $w$)
    \item Construct a CFG $H$ that generates $\Sigma^*$
    \item Feed $\langle G, H \rangle$ as input to the machine $R$
    \begin{itemize}
        \item If $R$ accepts, then $S$ rejects
        \item If $R$ rejects, then $S$ accepts
    \end{itemize}
\end{enumerate}
We have created a machine that decides $A_{TM}$, which is undecidable, using the assumed decidability of $R$. This is a contradiction so $R$ must be undecidable.

\section*{5.4}
(Not sure about answer)\\
$A = \{0^n1^n \mid n \geq 0\}$\\
$B = \{0^n \mid n \geq 0\}$\\
$f(x)= $ divide string in two, use first half\\
$A$ is not regular, $B$ is regular\\
$A \leq_m B$
\section*{5.9}

\begin{enumerate}
    \item Construct a TM T that decides $A$ = $\{ \langle M \rangle \mid w^R$ is accepted if w is accepted$\}$ inside a TM S used to decide $A_{TM}$.
    \item Inside S, construct a new TM $M_1$ from input $\langle \langle M\rangle, w \rangle$. Using input $w_1$:
    \begin{itemize}
        \item If $w_1 = w^R$, Accept
        \item Else if $w_1 = w$, Run $M$ on $w_1$
        \item Else, Reject 
    \end{itemize}
    \item Feed ${M_1}$ into TM T, if it accepts then $M$ must accept $w$ since T only accepts ${M}$ if both $w^R$ and $w$ are accepted and we can guarantee $w^R$ is accepted because we constructed $M_1$ to accept it.
\end{enumerate}
Thus we can use TM T to decide $A_{TM}$, which is undecidable, so T must also be undecidable.

\section*{5.22}

We let M be the recognizer of $A_{TM}$ and $f$ be the reduction from $A$ to $A_{TM}$. We describe a recognizer $N$ for $A$ as follows:
\begin{enumerate}
    \item $N$ = "On input w: 
    \begin{itemize}
        \item Compute $f(w)$
        \item Run M on $f(w)$ and output $M$'s outputs"
    \end{itemize}
\end{enumerate}

Clearly, if $w \in A$ then $f(w) \in A_{TM}$ because $f$ is a reduction from $A$ to $A_{TM}$. Thus, M accepts $f(w)$ whenever $w \in A$. Therefore, $N$ recognizes $A$.

\end{document}
