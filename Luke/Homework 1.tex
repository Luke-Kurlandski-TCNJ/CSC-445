\documentclass{article}
\usepackage[utf8]{inputenc}
\usepackage{relsize}
\usepackage{geometry}

\geometry{
 a4paper,
 total={170mm,257mm},
 left=20mm,
 top=20mm,
}

\title{Homework 1\\[0.2em]\smaller{}CSC 445-01: Theory of Computation}
\author{Matthew Mabrey, Luke Kurlandski}
\date{\today}

\begin{document}

\maketitle

\section*{Question I}

We will show that $A=B$ using three proofs by contradiction. First, it is helpful to expand the following
$$
A \times B = \biggr \{ (a,b) | a \in A, b \in B \biggr \}
$$
$$
B \times A = \biggr \{ (b,a) | b \in B, a \in A \biggr \}
$$
Thus the original condition is
$$
A \times B \subseteq B \times A
$$
$$
\biggr \{ (a,b) | a \in A, b \in B \biggr \} \subseteq \biggr \{ (b,a) | b \in B, a \in A \biggr \}
$$

\subsection*{1}
Suppose $A \not \subseteq B$. Then $\exists \; a \in A $ where $ (a \not \in B) $. Then $\exists \; (a, b) \in A \times  B $ where $ ((a, b) \not \in B \times  A ) $. This defies the original condition. We have proven
$$A \subseteq B$$

\subsection*{2}
Suppose $B \not \subseteq A$. Then $\exists \; b \in B $ where $ (b \not \in A) $. Then $\exists \; (a, b) \in A \times  B $ where $ ((a, b) \not \in B \times  A ) $. This defies the original condition. We have proven
$$B \subseteq A$$

\subsection*{3}
Suppose $A \neq B$. Then one of the following two statements must be true:
\begin{enumerate} 
\item $\exists \; a \in A $ where $ (a \not \in B) $, which violates $A \subseteq B$
\item $\exists \; b \in B $ where $ (b \not \in A) $, which violates $B \subseteq A$
\end{enumerate}
Thus we have proven
$$A = B$$


\section*{Question II}

\subsection*{a}

\subsection*{b}

\subsection*{c}

\section*{Question III}
We will prove
$$
P(n) \colon \sum^n_{i=1} \frac{1}{i(i+1)} = \frac{n}{n+1} \;\; \biggr | \;\; n \geq 0, n \in Z
$$
Base case:
\begin{eqnarray*}
P(0) &\colon& \sum^n_{i=1} \frac{1}{i(i+1)} = \frac{n}{n+1} \\
&\colon& \sum^0_{i=1} \frac{1}{i(i+1)} = \frac{0}{0+1} \\
&\colon& 0 = 0 \\
&\colon& \textrm{TRUE}
\end{eqnarray*}
Inductive Hypothesis:
$$
P(k) \colon \sum^k_{i=1} \frac{1}{i(i+1)} = \frac{k}{k+1} \;\; \biggr | \;\; k \geq 0, k \in Z
$$
Induction:
\begin{eqnarray*}
P(k+1) &\colon& \;\; \frac{(k+1)}{(k+1)+1} = \sum^{k+1}_{i=1} \frac{1}{i(i+1)}  \\
&\colon& \;\; \frac{k+1}{k+2} = \sum^k_{i=1} \frac{1}{i(i+1)} + \frac{1}{(k+1)((k+1)+1)} \\
&\colon& \;\; \frac{k+1}{k+2} = \frac{k}{k+1} + \frac{1}{(k+1)(k+2)} \\ 
&\colon& \;\; \frac{k+1}{k+2} = \frac{k(k+2) + 1}{(k+1)(k+2)} \\ 
&\colon& \;\; \frac{k+1}{k+2} = \frac{(k+1)^2}{(k+1)(k+2)} \\
&\colon& \;\; \frac{k+1}{k+2} = \frac{k+1}{(k+2)} \\
P(k+1) &\colon& \textrm{TRUE}
\end{eqnarray*}
Conclusion:
$$
\left (  P(0) \; \land \; \biggr ( P(k) \rightarrow P(k+1) \biggr ) \right ) \rightarrow P(n) 
$$

\section*{Question IV}

\subsection*{Truth Table}
\begin{displaymath}
\begin{array}{c c|c|c|c|c|c} % bars control columns
p & q & 
(p \rightarrow q) \land (p \rightarrow \neg q) & 
p \land (p \rightarrow q) & 
(p \rightarrow q) \land (\neg p \rightarrow q) &
p \iff (p \iff q) &
q \land (p \rightarrow q)
\\ 
\hline
T & T & F & T & T & T & T \\
T & F & F & F & F & F & F \\
F & T & T & F & T & T & T \\
F & F & T & F & F & F & F 
\end{array}
\end{displaymath}

\subsection*{Equivalence}
\begin{displaymath}
\begin{array}{c|c|c} % bars control columns
Problem & Original & Simplified \\
\hline
a & (p \rightarrow q) \land (p \rightarrow \neg q) & \neg p \\ 
b & p \land (p \rightarrow q) & p \land q \\
c & (p \rightarrow q) \land (\neg p \rightarrow q) & q \\
d & p \iff (p \iff q) & q \\
e & q \land (p \rightarrow q) & q
\end{array}
\end{displaymath}

\section*{Question V}

\subsection{1}

\subsection{2}

\subsection{3}

\section*{Question VI}

\subsection{1}

\subsection{2}

\section*{Question VII}

\subsection{1}

\subsection{2}

\end{document}
