
\documentclass{article}
\usepackage[utf8]{inputenc}
\usepackage{relsize}
\usepackage{amsmath}
\usepackage{geometry}
\usepackage{tikz}
\usetikzlibrary{automata, positioning, arrows}
\tikzset{node distance=2.5cm, % Minimum distance between two nodes. Change if necessary.
every state/.style={ % Sets the properties for each state
semithick,
fill=gray!10},
initial text={}, % No label on start arrow
double distance=2pt, % Adjust appearance of accept states
every edge/.style={ % Sets the properties for each transition
draw,->,>=stealth', % Makes edges directed with bold arrowheads
auto,semithick}}

\geometry{
 a4paper,
 total={170mm,257mm},
 left=20mm,
 top=20mm,
}

\title{Homework 5\\[0.2em]\smaller{}CSC 445-01: Theory of Computation}
\author{Matthew Mabrey, Luke Kurlandski}
\date{\today}

\begin{document}

\maketitle

\section*{2.15 Matthew}

\section*{2.30 Matthew}

We will prove L is not a context free language using a proof by contradiction and the Pumping Lemma.
$$L = \{ t_1\#t_2\#...\#t_k | k \geq 2, t_i \in \{ a, b\}^*, \exists ( t_i = t_j,  i \neq j ) \}$$
Suppose L is a CFL. Then for a string $s \in L$ of length greater than p, there exists some decomposition of s, $s = uvxyz$ such that 
\begin{enumerate}
    \item $uv^ixy^iz \in L$ for $i \geq 0$
    \item $|vy| > 0$
    \item $|vxy| \leq p$
\end{enumerate}
We let $s = $ because $|s| \geq p + 1$ and $s \in L$.\\\\

Ooooof

\section*{3.8}
\subsection*{b}
On input string $w$
\begin{enumerate}
    \item Move the head to the front of the tape. Scan the tape and mark the first 1 that has not been marked. If none are found, move to stage 5.
    \item Move the head to the front of the tape. Scan the tape and mark the first 0 that has not been marked. If no 0 is found, reject. 
    \item Scan the tape and mark the first 0 that has not been marked. If no 0 is found, reject.
    \item Go to stage 1
    \item Move the head to the front of the tape. Scan the tape to see if any unmarked 0s remain. If none remain, accept, else reject. 
\end{enumerate}

\subsection*{c}
Run the machine from part b on input w. If it accepts, then reject. If it rejects, then accept. Alternatively, build a new TM as described below.\\\\
On input string $w$
\begin{enumerate}
    \item Move the head to the front of the tape. Scan the tape and mark the first 1 that has not been marked. If none are found, move to stage 5.
    \item Move the head to the front of the tape. Scan the tape and mark the first 0 that has not been marked. If no 0 is found, accept. 
    \item Scan the tape and mark the first 0 that has not been marked. If no 0 is found, accept.
    \item Go to stage 1
    \item Move the head to the front of the tape. Scan the tape to see if any unmarked 0s remain. If none remain, reject, else accept. 
\end{enumerate}

\section*{3.15}

\subsection*{b Matthew}
Suppose we have Turing Machines $M_1$ and $M_2$ that decide languages $L_1$ and $L_2$ respectively. We describe the Turing Machine $M'$ that decides the concatenation of $L_1$ and $L_2$.\\\\
On input string $w$

\subsection*{c}
Suppose we have a Turing Machine $M$ that decides the language $L$. We describe the nondeterministic two-tape Turing Machine $M'$ that decides the language $L^*$.\\\\
On input string $w$
\begin{enumerate}
    \item Nondeterministically select an integer $k \leq |w| $
    \item Nondeterministically partition w into k components, $w_1, w_2, ..., w_k$
    \item Run M on each string $w_1, w_2, ..., w_k$
        \begin {itemize}
        		\item If M accepts all the strings $w_1, w_2, ..., w_k$ then accept
        \end{itemize}
\end{enumerate}

\subsection*{d}
Suppose we have a Turing Machine $M$ that decides the language $L$. We describe the Turing Machine $M'$ that decides the language $L^c$.\\\\
On input string $w$
\begin{enumerate}
    \item Run $M$ on w. 
    \begin{itemize}
        \item If $M$ accepts, then reject
        \item Else accept
    \end{itemize}
\end{enumerate}

\subsection*{e}
Having proved that decidable languages are closed under union and complement, DeMorgan's Law proves they are closed under intersection as well (as seen in other classes of languages).\\\\
Suppose we have Turing Machines $M_1$ and $M_2$ that decide languages $L_1$ and $L_2$ respectively. We describe the Turing Machine $M'$ that decides the intersection of $L_1$ and $L_2$.\\\\
On input string $w$
\begin{enumerate}
    \item Run $M_1$ on w 
    \begin{itemize}
        \item If $M_1$ accepts, continue to stage 2
        \item Else reject
    \end{itemize} 
    \item Run $M_2$ on w
    \begin{itemize}
        \item If $M_2$ accepts, accept
        \item Else reject
    \end{itemize}
\end{enumerate}

\end{document}
